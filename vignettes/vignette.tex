\documentclass{article}
\usepackage{url,Sweave}
%\VignetteIndexEntry{The micropan package vignette}

\title{The \texttt{micropan} package vignette}
\author{Lars Snipen and Kristian Hovde Liland}
\date{}

\begin{document}
\Sconcordance{concordance:vignette.tex:vignette.Rnw:%
1 67 1 1 2 4 0 1 2 6 1 1 2 4 0 1 2 8 1 1 2 4 0 1 2 9 1}

%\SweaveOpts{concordance=TRUE}

\maketitle


\section{Using \texttt{dplyr} and \texttt{stringr}}
A major change in the 2.0 version is the use of generic data structures and functions in R instead of creating package specific ones. This makes it possible to use the power of standard data manipulation tools and visualization that R-users are familiar with.

Compared to previous versions some functions have been moved to the \texttt{microseq} package.

You will also find no casestudy document or plotting functions. However, if you locate the GitHub site for this package, you will find a tutorial with code making similar plots using \texttt{ggplot} or \texttt{ggdendro}. This is an example of using generic R tools instead of making functions for each special case.

\subsection{Faster reading of BLAST results}
A major change in the 2.1 version is faster reading of the BLAST result files, see `?bDist` or the tutorial at GitHub mentioned above for more details. 


\section{External software}
Some functions in this package calls upons external software that must be available on the system. Some of these are 'installed' by simply downloading a binary executable that you put somewhere proper on your computer. To make such programs visible to R, you typically need to update your \texttt{PATH} environment variable, to specify where these executables are located. Try it out, and use google for help!


\subsection{Software \texttt{blast+}}
The function \emph{blastpAllAll} uses the free software \texttt{blast+} (ftp://ftp.ncbi.nlm.nih.gov/blast/executables/blast+/LATEST/). Source code and installers makes it straightforward to install. In the R console the command
\begin{Schunk}
\begin{Sinput}
> system("blastp -h")
\end{Sinput}
\end{Schunk}
should produce some sensible output.


\subsection{Software \texttt{hmmer}}
The functions \emph{hmmerScan()} uses the free software \texttt{hmmer} (http://hmmer.org/). This software is developed for UNIX systems (e.g. Mac or Linux), and Windows users may find it a little difficult to install and run from R. In the R console the command
\begin{Schunk}
\begin{Sinput}
> system("hmmscan -h")
\end{Sinput}
\end{Schunk}
should produce some sensible output.



\end{document}
